\documentclass[12pt,a4paper]{article}
\usepackage{cclicenses,graphicx}
\title{Tutorial on Tables and Figures}
\author{Author \\ email \\ \byncsa} % creative commons license
\date{\today}

\begin{document}
\maketitle
\newpage
\listoftables
\listoffigures
\newpage

\begin{tabular}{rr} % rr implies two columns that are right aligned
Mango & Mixed \\
Jack fruit & Kolli Hills \\ 
Banana & Green
\end{tabular}
\\.................................................................................\\
\begin{tabular}{|r|r|} % | line separates the columns by lines
Mango & Mixed \\
Jack fruit & Kolli Hills \\ 
Banana & Green
\end{tabular}
\\.................................................................................\\
\begin{tabular}{||r|r|}
Mango & Mixed \\
Jack fruit & Kolli Hills \\ 
Banana & Green
\end{tabular}
\\.................................................................................\\
\begin{tabular}{||l|c|} % left and centre aligned
Mango & Mixed \\
Jack fruit & Kolli Hills \\ 
Banana & Green
\end{tabular}
\\.................................................................................\\
\begin{tabular}{|c|c|}\hline % hline for horizontal lines (on top)
Mango & Mixed \\ 
Jack fruit & Kolli Hills \\ 
Banana & Green \\ 
\end{tabular}
\\.................................................................................\\
\begin{tabular}{|c|c|}\hline 
Mango & Mixed \\ \hline
Jack fruit & Kolli Hills \\ \hline
Banana & Green \\ \hline
\end{tabular}
\\.................................................................................\\
\begin{tabular}{||l|c|c|c|r|}\hline % last column has right align so that we can add the numbers
Fruit & Type & No. of units & cost/unit & cost (Rs.)\\ \hline
Mango & Mixed & 20 & 75 & 1,500 \\ \hline
Jack fruit & Kolli Hills & 10 & 50 & 500\\ \hline
Banana & Green & 10 & 20 & 200\\ \hline
\end{tabular}
\\.................................................................................\\
% multicolumn command
\begin{tabular}{||l|c|c|c|r|}\hline
\multicolumn{2}{c}{Fruit details} & \multicolumn{3}{c}{Cost calculations} \\ \hline % 2 implies 2 columns, c implies centre aligned
Fruit & Type & No. of units & cost/unit & cost (Rs.)\\ \hline
Mango & Mixed & 20 & 75 & 1,500 \\ \hline
Jack fruit & Kolli Hills & 10 & 50 & 500\\ \hline
Banana & Green & 10 & 20 & 200\\ \hline
\end{tabular}
\\.................................................................................\\
\begin{tabular}{||l|c|c|c|r|}\hline
\multicolumn{2}{||c|}{Fruit details} & \multicolumn{3}{c|}{Cost calculations} \\ \hline
Fruit & Type & No. of units & cost/unit & cost (Rs.)\\ \hline
Mango & Mixed & 20 & 75 & 1,500 \\ \hline
Jack fruit & Kolli Hills & 10 & 50 & 500\\ \hline
Banana & Green & 10 & 20 & 200\\ \hline
\end{tabular}
\\.................................................................................\\
% Drawing lines in the middle of columns 
\begin{tabular}{||l|c|c|c|r|}\hline
\multicolumn{2}{||c|}{Fruit details} & \multicolumn{3}{c|}{Cost calculations} \\ \hline
Fruit & Type & No. of units & cost/unit & cost (Rs.)\\ \hline
Mango & Malgoa & 18 & 50 & \\ \hline % we need a cline here now
		 & Alfonso & 2 & 300 & 1,500 \\ \hline
Jack fruit & Kolli Hills & 10 & 50 & 500\\ \hline
Banana & Green & 10 & 20 & 200\\ \hline
\end{tabular}
\\.................................................................................\\
\begin{tabular}{||l|c|c|c|r|}\hline
\multicolumn{2}{||c|}{Fruit details} & \multicolumn{3}{c|}{Cost calculations} \\ \hline
Fruit & Type & No. of units & cost/unit & cost (Rs.)\\ \hline
Mango & Malgoa & 18 & 50 & \\ \cline{2-4} % cline between 2 to 4 columns
		 & Alfonso & 2 & 300 & 1,500 \\ \hline
Jack fruit & Kolli Hills & 10 & 50 & 500\\ \hline
Banana & Green & 10 & 20 & 200\\ \hline
\end{tabular}
\\.................................................................................\\
\begin{tabular}{||l|c|c|c|r|}\hline
\multicolumn{2}{||c|}{Fruit details} & \multicolumn{3}{c|}{Cost calculations} \\ \hline
Fruit & Type & No. of units & cost/unit & cost (Rs.)\\ \hline
Mango & Malgoa & 18 & 50 & \\ \cline{2-4}
		 & Alfonso & 2 & 300 & 1,500 \\ \hline
Jack fruit & Kolli Hills & 10 & 50 & 500\\ \hline
Banana & Green & 10 & 20 & 200\\ \hline
\multicolumn{4}{||r|}{Total cost (Rs.)} & {2,200} \\ \hline
\end{tabular}
\\.................................................................................\\
\newpage
% placing the table in a table environment as an unit
This is a
\begin{table} % table environment
\centering % places the table at the center of the document
\caption{Cost of fruits in India} % adds caption to the table
\vspace{1ex} % adds space between the caption and the table
\begin{tabular}{||l|c|c|c|r|}\hline
\multicolumn{2}{||c|}{Fruit details} & \multicolumn{3}{c|}{Cost calculations} \\ \hline
Fruit & Type & No. of units & cost/unit & cost (Rs.)\\ \hline
Mango & Malgoa & 18 & 50 & \\ \cline{2-4}
		 & Alfonso & 2 & 300 & 1,500 \\ \hline
Jack fruit & Kolli Hills & 10 & 50 & 500\\ \hline
Banana & Green & 10 & 20 & 200\\ \hline
\multicolumn{4}{||r|}{Total cost (Rs.)} & {2,200} \\ \hline
\end{tabular}
\end{table}
an example table
\\.................................................................................\\
\newpage
% Adding references of tables
This is a
\begin{table} 
\centering 
\caption{Cost of fruits in India} 
% *always add the label after the caption command*
\label{tab:Fruits}
\vspace{1ex} 
\begin{tabular}{||l|c|c|c|r|}\hline
\multicolumn{2}{||c|}{Fruit details} & \multicolumn{3}{c|}{Cost calculations} \\ \hline
Fruit & Type & No. of units & cost/unit & cost (Rs.)\\ \hline
Mango & Malgoa & 18 & 50 & \\ \cline{2-4}
		 & Alfonso & 2 & 300 & 1,500 \\ \hline
Jack fruit & Kolli Hills & 10 & 50 & 500\\ \hline
Banana & Green & 10 & 20 & 200\\ \hline
\multicolumn{4}{||r|}{Total cost (Rs.)} & {2,200} \\ \hline
\end{tabular}
\end{table}
an example table

The cost of fruits is shown in Table \ref{tab:Fruits}

\newpage
% Including figures in a report, use graphicx package
\begin{figure}
\centering
\includegraphics[width=0.5\linewidth]{iitb} % give the pdf with figure here, half line width is used
% *figure captions are to be given after the figure is included (above command)*
\caption{Golden jubilee logo of IIT Bombay}
\label{fig:golden}
\end{figure}

Text figure \ref{fig:golden}

% rotating figures
\begin{figure}
\centering
\includegraphics[width=0.5\linewidth,angle=90]{iitb} % angle argument for rotating figures
\caption{Golden jubilee logo of IIT Bombay}
\label{fig:golden1}
\end{figure}
Text figure \ref{fig:golden1}


\end{document}
