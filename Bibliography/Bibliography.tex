\documentclass[12pt]{article}
\usepackage{amsmath,cclicenses}
\usepackage{harvard} % this is a downloaded package for harvard style of referencing (a .sty file)
\title{Tutorial on Bibliography}
\author{Author\\email\\ \byncsa}
\date{\today}
\bibliographystyle{ifac} % The style of bibliography to be used
% {plain} : plain bibliography
% {unsrt} : The references are ordered according to the sequence that appears first
% {alpha} : Style used by computer scientists
% *when using the harvard style* use {ifac} a downloaded .bst file


\begin{document}
\maketitle
\newpage
\section{Aryabatta's Identity for Control Design}

Polynomial equations of the form
\begin{align*}
X(z)+D(z) + Y(z)N(z) = C(z)
\end{align*}
arise frequently in control system design. In the above equation, $D(z)$, $N(z)$ and $C(z)$ are known polynomials and $X(z)$ and $Y(z)$ are unknowns, to be determined. This equation is known as Diophatine equation \cite{vk79,tk80} and Aryabatta's identity \cite{mv85}. A solution technique to this identity is presented in
\cite{cp82}.  Matlab and Scilab implementations of this solution are
available on the web \cite{kmm1-07}.

% cite as noun *works only for harvard style* of referencing, where the name of reference appears without the brackets
The textbook by \citeasnoun{kmm07} 
illustrates several control design
examples using Aryabhatta's identity.  The approach followed in this
book is explained in \cite{ms04,km06}.  In addition to handling
control design problems in conventional domains, this approach will 
be useful also for naturally discrete time problems that arise in
computing systems, see for example, \cite{mmr03,mrbm04,vs06}.

\bibliography{ref} % This tells LaTeX the file name in which the references can be found
\end{document}

% Procedure for adding references
% 1. Make the source as above
% 2. In command line, go the folder containing the *.tex and **.bib file (where *,** are to be replaced with your respective filenames) 
% replace * with your filename in the following commands
% 3. run `pdflatex *'
% 4. run `bibtex *'
% 5. run `pdflatex *' twice