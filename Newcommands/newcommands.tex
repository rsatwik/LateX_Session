% for defining new command
% \newcommand{command}{definition} or
% \newcommand{command}[parameters]{definition}
% at the beginning of the document

\documentclass{article}
\usepackage{amsfonts}

% Example 1
%\newcommand{\bbr}{\mathbb R} % type set R in mathbb i.e. math black board font 
%\begin{document}
%Let $\bbr$ be a set of Rational numbers. % using the new command
%\end{document}

% Example 2
%\newcommand{\bb}[1]{\mathbb {#1}} % 1 : the commands has 1 parameter
%\begin{document}
%Let $\bb{R}$ be a set of Rational numbers and $\bb{Z}$ be the set of Complex numbers
%\end{document}

% Example 3
%\newcommand{\add}[2]{\left(#1+#2\right)}
%\begin{document}
%Adding abc and xyz we get
%$\add{abc}{xyz}$
%\end{document}

% Example 4
\newcommand{\textbfit}[1]{\textbf{\emph{#1}}}
\begin{document}
Adding abc and xyz we get $\textbfit{abc*xyz}$ as the result
\end{document}

%..................................................................................................

% overloading an existing command
% \renewcommand{command}{definition}
% at the beginning of the document

% Example 1
%\renewcommand{\S}{\mathcal{S}} % we redefined \S to produce a calligraphic font instead of section symbol
%\begin{document}
%Let $\S$ be a set
%\end{document}