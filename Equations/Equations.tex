\documentclass[12pt]{article}
\usepackage[textwidth=15cm]{geometry}
\usepackage{amsmath}
\usepackage{mathtools}

\begin{document}
The following proportional, derivative controller has two tuning parameters: $K$ and $\tau_d$

%
% First equation
%
\begin{align*}  % align plays the role of $ mode and hence $ is avoided in the following equations
u(t) &= K \left[ e(t)+\tau_d\frac{de(t)}{dt}\right] % \left[ and \right] creates square brackets
% place all the equations in one align to align all the equations
%
% &= aligns the equations by = sign
%
% Second equation
%
\intertext{We want to apply the above controller to the following equation:} % for text in align mode
% when in align never use blank lines, use % to create blank lines
%
% LHS
%
\frac{d}{dt}
\begin{bmatrix}
x_1\\x_2\\x_3
\end{bmatrix} &=
%
% RHS
%
\begin{bmatrix}
0 & 0 & 1 \\
0 & 0 & 0 \\
0 & \alpha & 0 \\
\end{bmatrix}
\begin{bmatrix}
x_1\\x_2\\x_3
\end{bmatrix}+
%
% || Term
%
\begin{bmatrix}
0\\0\\ \beta
\end{bmatrix}\mu \\ %\\ for getting equations one below another
%
% Third equation
%
% when in align never use blank lines, use % to create blank lines
%
% LHS
%
\frac{d}{dt}
\begin{bmatrix}
x_1\\x_2\\x_3
\end{bmatrix} &=
%
% RHS
%
\begin{bmatrix}
0 & 0 & 1 \\
0 & 0 & 0 \\
0 & \alpha & 0 \\
\end{bmatrix}
\begin{bmatrix}
x_1\\x_2\\x_3
\end{bmatrix}+
%
% || Term
%
\begin{bmatrix}
0\\0\\ \beta
\end{bmatrix}\mu
\end{align*}
The above equation is the model of a plant.\\

\begin{align*}
a &= b+c \\
a\,b\,c\,d &= f + g + h
\end{align*}

\begin{align*}
& a = b+c \\  % & is used to align the text differently
& a\,b\,c\,d = f + g + h
\end{align*}

\begin{align*}
\alpha &= \beta + \gamma \\
\alpha + \beta &= \frac{\gamma}{\delta} + \delta\int\mu d \mu \\
\alpha + \beta \mu &= \gamma \delta \\
\end{align*}

\newpage
%
% Numbering the equations
%
\begin{align}  % remove * in align* to number the equations
u(t) &= K \left[ e(t)+\tau_d\frac{de(t)}{dt}\right] 
%
% Second equation
%
\intertext{We want to apply the above controller to the following equation:}
%
% LHS
%
\frac{d}{dt}
\begin{bmatrix}
x_1\\x_2\\x_3
\end{bmatrix} &=
%
% RHS
%
\begin{bmatrix}
0 & 0 & 1 \\
0 & 0 & 0 \\
0 & \alpha & 0 \\
\end{bmatrix}
\begin{bmatrix}
x_1\\x_2\\x_3
\end{bmatrix}+
%
% || Term
%
\begin{bmatrix}
0\\0\\ \beta
\end{bmatrix}\mu
\end{align}
The above equation 2 is the model of a plant.\\





%
% Not Numbering certain equations
%
\begin{align} 
u(t) &= K \left[ e(t)+\tau_d\frac{de(t)}{dt}\right]\nonumber % \left[ and \right] creates square brackets
% 
% Second equation
%
\intertext{We want to apply the above controller to the following equation:}
%
% LHS
%
\frac{d}{dt}
\begin{bmatrix}
x_1\\x_2\\x_3
\end{bmatrix} &=
%
% RHS
%
\begin{bmatrix}
0 & 0 & 1 \\
0 & 0 & 0 \\
0 & \alpha & 0 \\
\end{bmatrix}
\begin{bmatrix}
x_1\\x_2\\x_3
\end{bmatrix}+
%
% || Term
%
\begin{bmatrix}
0\\0\\ \beta
\end{bmatrix}\mu
\label{PID} % label and reference links the equation to the text. Further, when the equation is modified the text is modified automatically. *Requires twice compilation*.
\end{align}
The above equation \ref{PID} is the model of a plant.\\

\begin{align*}|x| =\begin{cases}x & \text{if $x\ge 0$}\\-x & \text{if $x\le 0$}\end{cases}\end{align*}


This example shows \verb|aligned| equations within
an \verb|align| environment.
\begin{align}
  \phantom{i + j + k}
  &\begin{aligned}
    \mathllap{a} &= b + c + d\\
      &\qquad + e + f + g + x + y + z
  \end{aligned}\\
  &\begin{aligned}
    \mathllap{i + j + k} &= l + m + n\\
      &\qquad + o + p + q
  \end{aligned}
\end{align}

\end{document}